The adaptive policy manages the number of GC threads 
at every collection cycle using two key factors.
There is a heuristic assumption that the number of GC threads should be
at least double the number of mutator threads, $T_\mathrm{mutator}$.
Further, a GC thread should process 64MB of data on average.
Consequently, the initial number of GC threads $T_\mathrm{GC}^0$
is calculated as:

\begin{equation}
T_\mathrm{GC}^0 = \mathrm{max} \big( 2T_\mathrm{mutator}, \frac{\mathrm{heapsize}}{64MB} \big)
\end{equation}
For subsequent GCs, we also consider the number of GC threads at the previous GC. The number of threads for the $n$th GC, $T_\mathrm{GC}^n$ is
calculated as:

\begin{equation}
T_\mathrm{GC}^n = \frac{\mathrm{max} \big( 2T_\mathrm{mutator}, \frac{\mathrm{heapsize}}{64MB} \big) + T_\mathrm{GC}^{n-1}}{2}
\end{equation}
The rate of change in GC thread count is asymmetric.
When the runtime decides to increase the number of threads, it does so 
instantly. 
However, decreasing the number of GC threads is done gradually.
Figure \ref{fig:policy} shows the adaptive policy schematically.
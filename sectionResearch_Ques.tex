\section{Research Questions}
\label{sec:rquestions}

This section identifies the three research questions 
that we aim to answer systematically and quantitatively in this
study. The first question relates to the observed behavior of the 
existing runtime platform. The final two questions relate to 
performance optimization, based on our
modifications to the runtime platform.

\textbf{RQ1: Does GC multi-threading cause NUMA congestion?}

Section \ref{sec:congestion} deals with RQ1.
Production high-performance VMs feature parallel GC implementations.
On NUMA machines, it is likely that heap regions are spread across 
the NUMA nodes. GC threads will access memory from remote nodes.
This can cause NUMA congestion, if inter-node data traffic
bandwidth is too large.

\textbf{RQ2: Does a statically optimum thread configuration improve GC performance significantly, in relation to the default configuration?}

Section \ref{sec:staticopt} explores RQ2.
In this study, we mean \emph{throughput} whenever we refer to performance.
The default configuration is the out-of-the-box GC threading policy.
Here, \emph{statically optimal} means we use an ahead-of-time tuning pass
to determine the best configuration. This configuration may be specialized
for the platform, for a particular benchmark and even for an
input data set to that benchmark \cite{mao09influence}.

\textbf{RQ3: Does an adaptive thread management policy 
improve GC performance significantly, 
in relation to the default configuration?}

Section \ref{sec:dynamicopt} deals with RQ3.
This question is similar to RQ2, except that we dynamically vary the
GC configuration. We use a simple search-based optimization technique, 
varying the number of GC threads at runtime to optimize 
measured throughput.